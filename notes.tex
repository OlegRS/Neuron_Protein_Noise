\documentclass[a4paper, 11pt]{article}

\usepackage[top=112pt, bottom=112pt, left=90pt, right=85pt]{geometry}
\usepackage{geometry, amssymb, csquotes, amsmath, graphicx, mathtools, amsthm, calc, accents} 
\usepackage[hidelinks]{hyperref}
\usepackage[utf8]{inputenc}
\usepackage{xcolor}
%\usepackage{parskip}

\newcommand{\doubletilde}[1]{\tilde{\raisebox{0pt}[0.85\height]{$\tilde{#1}$}}}
\newcommand{\tripletilde}[1]{\tilde{\raisebox{0pt}[0.85\height]{$\doubletilde{#1}$}}}

\newtheorem{prop}{Proposition}
\newtheorem{definition}{Definition}
\newtheorem{lemma}{Lemma}
\newtheorem{theor}{Theorem}
\newtheorem{cor}{Corollary}
\newtheorem{conjecture}{Conjecture}

\DeclareMathOperator\artanh{artanh}
\DeclareMathOperator\tr{tr}

\hypersetup{
  urlcolor = black,
  citecolor = black,
  pdftitle = {notes},
  pdfsubject = {notes},
  pdfpagemode = UseNone
}

% \hypersetup{
%     colorlinks=true,
%     linkcolor=blue,
%     filecolor=magenta,      
%     urlcolor=blue,
% }

\newcommand\underl[2]{\mathrel{\mathop{#2}\limits_{#1}}}

\title{\textbf{Master equations}}
\date{\today}
\begin{document}
\maketitle

\begin{abstract}
\end{abstract}

\tableofcontents

\section{Master equation from \cite{PAULSSON2005157}}

The master equation (ME) from \cite{PAULSSON2005157} reads as
\begin{equation}
  \begin{split} \label{full_Paulsson_ME}
    \frac{dP_{n_1,n_2,n_3}(t)}{dt} = &\lambda_1^+\left[(N_1-n_1+1)P_{n_1-1,n_2,n_3}(t) - (N_1-n_1)P_{n_1,n_2,n_3}(t)\right]\\
    + & \lambda_1^-\left[(n_1+1)P_{n_1+1,n_2,n_3}(t) - n_1P_{n_1,n_2,n_3}(t)\right] \\
    + & \lambda_2\left[n_1P_{n_1,n_2-1,n_3}(t) - n_1P_{n_1,n_2,n_3}(t)\right]\\
    + & \tau_2^{-1}\left[(n_2+1)P_{n_1,n_2+1,n_3}(t) - n_2P_{n_1,n_2,n_3}(t)\right]\\
    + & \lambda_3\left[n_2P_{n_1,n_2,n_3-1}(t) - n_2P_{n_1,n_2,n_3}(t)\right]\\
    + & \tau_3^{-1}\left[(n_3+1)P_{n_1,n_2,n_3+1}(t) - n_3P_{n_1,n_2,n_3}(t)\right]
  \end{split}
\end{equation}
and takes into account transcription switching of $N_1$ independent copies of the same gene ($n_1$ is the number of active replicas of the gene at any given time), mRNA production and decay ($n_2$ is the number of mRNA molecules), and the production and decay of proteins ($n_3$ is the number of proteins).

\section{Independent gene switching} \label{sec::independent_gene_switching}

We will now demonstrate the approach by solving the problem of independent switching of gene replicas, for which the master equation is given by\footnote{Compare with the first two lines of (\ref{full_Paulsson_ME}).}
\begin{equation}
  \begin{split} \label{gene_switching_ME}
    \frac{dP_{n_1}(t)}{dt} = &\lambda_1^+\left[(N_1-n_1+1)P_{n_1-1}(t) - (N_1-n_1)P_{n_1}(t)\right]\\
    + & \lambda_1^-\left[(n_1+1)P_{n_1+1}(t) - n_1P_{n_1}(t)\right],
  \end{split}
\end{equation}
where $n_1$ is the number of active genes.

The generating function of the distribution $P_{n_1}(t)$ is defined as
\begin{equation} \label{gen_func_def} 
  G(z_1; t) = \sum_{n_1=1}^{\infty}P_{n_1}(t)z_1^{n_1},
\end{equation}
and allows recasting the master equation (\ref{gene_switching_ME}) in a PDE form as follows.

First, we multiply both sides of the ME (\ref{gene_switching_ME}) by $z_1^{n_1}$ and take a sum over $n_1$
\begin{equation}
  \begin{split}
    \sum_{n_1=1}^{\infty}\frac{dP_{n_1}(t)}{dt}z_1^{n_1} = &\lambda_1^+\sum_{n_1=1}^{\infty}\left[(N_1-n_1+1)P_{n_1-1}(t) - (N_1-n_1)P_{n_1}(t)\right]z_1^{n_1}\\
    + & \lambda_1^-\sum_{n_1=1}^{\infty}\left[(n_1+1)P_{n_1+1}(t) - n_1P_{n_1}(t)\right]z_1^{n_1}.
  \end{split}
\end{equation}
Then, using the definition of the generating function (\ref{gen_func_def}) and the following identities
\begin{align*}
  & \sum_{n_1=1}^{\infty}P_{n_1-1}(t)z_1^{n_1} = z_1\sum_{n_1=1}^{\infty}P_{n_1-1}(t)z_1^{n_1-1} = z_1G(z_1;t)\\
  & \sum_{n_1=1}^{\infty}P_{n_1+1}(t)z_1^{n_1} = \frac{1}{z_1}\sum_{n_1=1}^{\infty}P_{n_1+1}(t)z_1^{n_1+1} = \frac{1}{z_1}G(z_1;t)\\
  & \sum_{n_1=1}^{\infty}n_1P_{n_1}(t)z_1^{n_1} = z_1\frac{\partial}{\partial z_1}\sum_{n_1=1}^{\infty}P_{n_1}(t)z_1^{n_1} = z_1\frac{\partial G(z_1; t)}{\partial z_1}\\
  & \sum_{n_1=1}^{\infty}n_1P_{n_1-1}(t)z_1^{n_1} = z_1\frac{\partial}{\partial z_1}\sum_{n_1=1}^{\infty}P_{n_1-1}(t)z_1^{n_1} = z_1\frac{\partial}{\partial z_1} \left(z_1G(z_1; t)\right)\\
  & \sum_{n_1=1}^{\infty}n_1P_{n_1+1}(t)z_1^{n_1} = z_1\frac{\partial}{\partial z_1}\sum_{n_1=1}^{\infty}P_{n_1+1}(t)z_1^{n_1} = z_1\frac{\partial}{\partial z_1} \left(\frac{1}{z_1}G(z_1; t)\right)
\end{align*}

Note that the moments of the probability distribution $P_{n_1}$ can be found from the derivatives of the generating function with respect to the corresponding variables, i.e.
\begin{align}
  \frac{\partial G(z_1; t)}{\partial z_1}\bigg\rvert_{z_1=1} &= \sum_{n_1=1}^{\infty} n_1P_{n_1}(t) = \langle n_1\rangle(t);\\
  \frac{\partial^2 G(z_1; t)}{\partial z_1^2}\bigg\rvert_{z_1=1} &= \sum_{n_1=1}^{\infty} n_1(n_1-1)P_{n_1}(t) = \langle n_1\rangle^2(t) - \langle n_1\rangle(t).
\end{align}

For the generating function, the equation (\ref{gene_switching_ME}) gives the following first-order partial differential equation (PDE)
\begin{equation} \label{gene_switching_PDE}
  \frac{\partial G(z_1;t)}{\partial t} + (\lambda_1^+z_1 + \lambda_1^-)(z_1-1)\frac{\partial G(z_1;t)}{\partial z_1} = \lambda_1^+N_1(z_1-1)G(z_1;t),
\end{equation}
which can be solved using the method of characteristics.

For convenience we introduce new variable as
\begin{equation*}
  x_1 := z_1 - 1
\end{equation*}
and denote
\begin{equation}
  \mathcal G(x_1; t) = G(z_1(x_1); t).
\end{equation}
in terms of which the PDE (\ref{gene_switching_PDE}) becomes
\begin{equation} \label{gene_switching_PDE_x}
  \frac{\partial \mathcal G(x_1;t)}{\partial t} + (\lambda_1^+x_1 + \frac{1}{\tau_1})x_1\frac{\partial \mathcal G(x_1;t)}{\partial x_1} = \lambda_1^+N_1x_1\mathcal G(x_1;t),
\end{equation}
where $\tau_1 := (\lambda_1^+ + \lambda_1^-)^{-1}$.
The characteristic equations for the nonhomogeneous quasilinear PDE (\ref{gene_switching_PDE}) are given by
\begin{equation*}
  \begin{dcases}
    \dot t = \tau;\\
    \dot x_1 = \lambda_1^+x_1^2 + \frac{1}{\tau_1}x_1;\\
    \dot G = \lambda_1^+N_1x_1G,
  \end{dcases}
\end{equation*}
which have the following {\it first integrals}
\begin{align*}
  C_1& = -\left(P_\text{on}+\frac{1}{x_1}\right)\mathrm e^{t/\tau_1}\\
  C_2& = \left(\frac{-\mathrm e^{t/\tau_1}}{x_1P_{on}}\right)^{N_1}G,
\end{align*}
where
\begin{equation}
  P_\text{on} = \frac{\lambda_1^+}{\lambda_1^+ + \lambda_1^-} = \lambda_1^+\tau_1.
\end{equation}

The general solution is given by\footnote{As an exercise one can check this solution by direct substitution to (\ref{gene_switching_PDE_x}).}
\begin{equation}\label{n1_general_solution}
  \mathcal G(x_1;t) = \left(-x_1P_{\text{on}}\mathrm e^{-t/\tau_1}\right)^{N_1}f\left(\left(P_{on}+\frac{1}{x_1}\right)\mathrm e^{t/\tau_1}\right),
\end{equation}
where $f(.)$ is an arbitrary function defined by the initial conditions.

Assuming that the initial number of active genes was $m$, i.e.
\begin{equation}\label{initial_condition}
  P_{n_1}(t=0) = \delta_{n_1,m} \implies \mathcal G(x_1; t=0) = (x_1+1)^m,
\end{equation}
we arrive to
\begin{equation}
  f(\xi) = \left(\frac{\xi - P_\text{on}}{-P_{\text{on}}}\right)^{N_1}\cdot\left(\frac{1+\xi-P_\text{on}}{\xi-P_{\text{on}}}\right)^m,
\end{equation}
from which (\ref{n1_general_solution}) reads as\footnote{Which, of course, can also be checked by the direct substitution to the PDE (\ref{gene_switching_PDE_x}) and the initial condition (\ref{initial_condition}).}
\begin{equation}\label{n1_special_solution}
  \mathcal G(x_1; t) = \left(x_1P_\text{on}\left(1-\mathrm e^{-t/\tau_1}\right)+1\right)^{N_1}\cdot\left(\frac{x_1\mathrm e^{-t/\tau_1}}{x_1P_\text{on}\left(1-\mathrm e^{-t/\tau_1}\right)+1} + 1\right)^m.
\end{equation}

%% \begin{equation}
%%   \mathcal G(x_1; t) = \left(x_1P_{\text{on}}\left(1-\mathrm e^{-t/\tau_1}\right)\right),
%% \end{equation}
%% and finally

Note that, as any generating function, $\mathcal G(x_1=0; t)=1$, and from (\ref{n1_special_solution}) it is clear that
\begin{equation}\label{binomial_gen_function}
  \lim_{t\to\infty}\mathcal G(x_1; t) = \left(1 + x_1P_\text{on}\right)^{N_1},
\end{equation}
which is expected, since (\ref{binomial_gen_function}) is the generating function of the binomial distribution with the probability of success $P_\text{on}$.

\subsection{Asymptotic expansion of the stationary solution}

\section{Compartment hopping}
We will now consider the problem of hopping between two spatial compartments, which is mathematically equivalent to the problem from Section \ref{sec::independent_gene_switching}, but we will approach it differently. Suppose the number of molecules in one compartment (say, soma) is $m_0$, and in another (say, dendrite) is $m_1$. The events of hopping are independent, and the ME for the process is given by
\begin{equation}
  \begin{split}
    \frac{dP_{m_0, m_1}(t)}{dt} = &\nu^+\left[(m_0+1)P_{m_0+1,m_1-1}(t)-m_0P_{m_0,m_1}(t)\right]\\ + &\nu^-\left[(m_1+1)P_{m_0-1, m_1+1}(t) - m_1P_{m_0, m_1}(t)\right].
  \end{split}
\end{equation}

The corresponding PDE for the generating function is given by
\begin{equation}\label{hopping_PDE}
  \frac{\partial\mathcal G(\mathbf x; t)}{\partial t} + \nu^+(x_0-x_1)\frac{\partial\mathcal G(\mathbf x; t)}{\partial x_0} + \nu^-(x_1-x_0)\frac{\partial\mathcal G(\mathbf x; t)}{\partial x_1} = 0,
\end{equation}
which we solve using the method of characteristics.

The general solution of the PDE (\ref{hopping_PDE}) is given by
\begin{equation}
  \mathcal G(\mathbf x; t) = f\left(\frac{x_0-x_1}{\nu^++\nu^-}\mathrm e^{-(\nu^-+\nu^+)t}, x_0\nu^-+\nu^+x_1\right),
\end{equation}
where $f(.,.)$ is an arbitrary function defined by the initial conditions.

If we assume that the total number of molecules is given by $M$ and at $t=0$ there was $n$ proteins in the soma (i.e. $M-n$ molecules in the dendrite), the initial probability distribution reads as
\begin{equation}
  P_{m_0, m_1}(t=0) = \delta_{n,m_0}\delta_{M-n,m_1},
\end{equation}
which has the following generating function
\begin{equation}\label{hopping_init_cond}
  \mathcal G(\mathbf x; t=0) = (x_0+1)^n(x_1+1)^{M-n}.
\end{equation}

The solution of the PDE (\ref{hopping_PDE}) with the initial condition (\ref{hopping_init_cond}) is given by
\begin{equation}
  \begin{split}
    \mathcal G(\mathbf x; t) = &\left[x_0+1+P_{on}\left(1-\mathrm e^{-\frac{t}{\tau}}(x_1-x_0)\right)\right]^n\\
    &\cdot\left[x_0+1+(x_1-x_0)\left(P_{on}+\mathrm e^{-\frac{t}{\tau}}(1-P_{on})\right)\right]^{M-n},
  \end{split}
\end{equation}
where $\tau := (\nu^++\nu^-)^{-1}$ and $P_{on} = \nu^+\tau$.
\subsection{Asymptotic expansion of the stationary solution}

\section{Full master equation (\ref{full_Paulsson_ME})}
Using the method as we did in Section \ref{sec::independent_gene_switching}, the PDE for the generating function is found to be
\begin{equation*} \label{full_Paulsson_PDE}
  \begin{split}
    \frac{\partial G}{\partial t} + \left[\lambda_1^+z_1(z_1-1) + \lambda_1^-(z_1-1) - \lambda_2(z_2-1)z_1\right]\frac{\partial G}{\partial z_1}\\ + \left[\frac{1}{\tau_2}(z_2-1) - \lambda_3(z_3-1)z_2\right]\frac{\partial G}{\partial z_2} + \frac{1}{\tau_3}(z_3-1)\frac{\partial G}{\partial z_3} = \lambda_1^+N_1(z_1-1)G,
  \end{split}
\end{equation*}
or, in terms of $\mathbf x = \mathbf z - \mathbf 1$ and $\mathcal G$
\begin{equation} \label{full_Paulsson_PDE_x}
  \begin{split}
    \frac{\partial \mathcal G}{\partial t} + \left[(\lambda_1^+x_1+\frac{1}{\tau_1})x_1  - \lambda_2x_2(x_1+1)\right]\frac{\partial \mathcal G}{\partial x_1}\\ + \left[\frac{1}{\tau_2}x_2 - \lambda_3x_3(x_2+1)\right]\frac{\partial \mathcal G}{\partial x_2} + \frac{1}{\tau_3}x_3\frac{\partial \mathcal G}{\partial x_3} = \lambda_1^+N_1x_1\mathcal G,
  \end{split}
\end{equation}
where $\tau_1 = ()$

Characteristic equations are given by
\begin{equation} \label{full_characteristics}
  \begin{dcases}
    \dot t = 1\\
    \dot x_1 = \lambda_1^+(x_1+1)x_1 + \lambda_1^-x_1 - \lambda_2x_2(x_1+1)\\
    \dot x_2 = \frac{1}{\tau_2}x_2 - \lambda_3x_3(x_2+1)\\
    \dot x_3 = \frac{1}{\tau_3}x_3\\
    \dot {\mathcal G} = \lambda_1^+N_1x_1\mathcal G
  \end{dcases}
\end{equation}
and we need to find its first integrals $\{C_i(\mathbf x; t; \mathcal G)|i=1,2,3,4\}$.

After using the first characteristic equation and setting $t = \xi$ (where $\xi$ is the parameter of the characteristics) the system of ODEs (\ref{full_characteristics}) can be approached by solving the 4-th ODE for $x_3$, which gives
\begin{equation}
  x_3 = C_1\mathrm e^{t/\tau_3} \left(\implies C_1(\mathbf x; t; \mathcal G) = x_3e^{-t/\tau_3}\right),
\end{equation}
then substituting the result into the ODE for $x_2$, which gives
\begin{equation}
  \begin{split}
    x_2 = -\lambda_3C_1\exp\left(\frac{t}{\tau_2}-\lambda_3C_1\tau_3\mathrm e^{t/\tau_3}\right)\int\exp\left[t\left(\frac{1}{\tau_3} - \frac{1}{\tau_2}\right) + \lambda_3C_1\tau_3\mathrm e^{t/\tau_3}\right]dt\\ + C_2\exp\left(\frac{t}{\tau_2}-\lambda_3C_1\tau_3\mathrm e^{t/\tau_3}\right)
  \end{split}
\end{equation}
then solving the second ODE for $x_1$, and then substituting $x_1$ to the last characteristic equation, which can be solved by separation.

Unfortunately, it seems that the expressions are too complicated to obtain the first integrals in a manageable way. Therefore, we need some simplifications/approximations.

\subsection{Simplifications}

The first simplification is to find the stationary distribution.

\textcolor{blue}{The second simplification is to linearise the coefficient of the PDE (\ref{full_Paulsson_PDE_x}) about $\mathbf x = 0$ since we only need to know the behaviour (the derivatives) of the generating function in the viscinity of $\mathbf x = 0$. Unfortunately, such linearisation seems to be insufficient to find second and higher moments of the distribution $P_{\mathbf n}(t)$.} \textcolor{red}{I think, the expression (4) from \cite{PAULSSON2005157} was obtained by constructing the asymptotic expansion (perturbation theory) for $\mathcal G(\mathbf x; t\to\infty)$ near $\mathbf x=\mathbf 0$ up to the second order in $\mathbf x$.}

\subsection{Asymptotic expansion of the stationary solution}



\section{Master equation for the simple neuron model}
\textcolor{blue}{For the simple neuron model with several spatial compartments we may probably assume that $N_1=1$ (there is only one copy of the gene encoding the protein of interest).}


\bibliographystyle{unsrt}
\bibliography{bibliography.bib}

\end{document}
